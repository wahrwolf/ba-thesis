Übersetzungen müssen schnell und kostengünstig sein.
Herkömliche Übersetzungen sind zeitintensiv und insgesamt eher teuer.
Machischnelle Übersetzungssysteme wurden deshalb in letzer Zeit immer beliebeter und sind inzwischen allgemein akzetiert.
Neuronale Maschinenübersetzungen bezeichnen eine Untergruppe der machinellen Übersetzung, die bereits heute bemerkenswert gute Ergebenisse erzielt und nach aktueller Erkenntis sogar noch Verbesserungspotential hat.
Eine der am häufigsten diskutierten Optimisierungsstrategien beschreibt die Verwendung von "Prefix Constraints" als Domän Adaptionmechanismus.
Die Auswirkungen der Verwendung dieses Mechanismus auf die Übersetzungsqualität ist allerdigns nicht hinreichend gut untersucht.

Für drei Bereiche (Recht, Finanzen, Medizin) wurde jeweils ein offener Korpus mit Deutsch-Englisch als verwandtes und Tschechisch-Englisch als nicht verwandtes Sprachpaar verwendet.
Die Datenvorverarbeitung beinhaltete die Zusammenführung der 3 domänenspezifischen Korpora zu einem Multi-Domain-Korpus, die Datenreduktion durch Aufteilung des Korpus in kleine(re) logische Einheiten und das Erstellen eines Multidomain-Korpus durch Rekombination diesr, die Kodierung in häufige Subsequenzen zur Reduzierung der Anzahl der Token und die Anwendung von des Domänadaptionsmechanismus "Prefix-Constraints", was zu 4 Multi-Domain-Korpora führte.
Drei verschiedene Bewertungsmetrken (BLEU zur Messung der Übersetzungspräzision, METEOR zur Beurteilung der Verständlichkeit und ROUGE zur Bewertung der Domänenspezialisierung) wurden verwendet, um die Auswirkungen des Adaptionsmechansimus auf die Übersetzungsleistung des NMT unter der Berücksichting des Verwandheit der Sprachpaar untereinander zu bewerten.
Die Auswertung ergab, dass die Verwendung des Mechansimus "Prefix Constraints" die Fähigkeit zur Erkennung und Reproduktion von Gemeinsamkeiten auf Kosten der Reprouktionsfähigkeit von Domän-Jargons verbessert.
Der Sprachvergleich ergab, dass sich die Übersetzungsqualität (indiziert durch die ausgewählten 3 Bewerungsmetriken) im verwandten (De-En) im Vergleich zum nicht verwandten Sprachpaar (Cz-En) unterschieden.
Dies könnte auf sprachbedingte Unterschiede in den Domänen zurückzuführen sein.
Weitere Forschung ist erforderlich, um den potenziellen Nutzen der Verwendung der "Prefx Constraints" ein bestimmtes Sprachpaar vorherzusagen.
Besondere Aufmerksamkeit sollte der Ausprägung sprachbedingter Unterschiede zwischen Domänen geschenkt werden.
